\documentclass{article}
\usepackage[utf8]{inputenc}
\usepackage[14pt]{extsizes}
\usepackage[english, russian]{babel}
\usepackage{setspace,amsmath}
\usepackage{listings}
\usepackage{xcolor}
\usepackage{hyperref}
\usepackage[left=20mm, top=15mm, right=15mm, bottom=15mm, nohead, footskip=10mm]{geometry}

\definecolor{linkcolor}{HTML}{799B03} 
\definecolor{urlcolor}{HTML}{FF0000} 

\hypersetup{pdfstartview=FitH,  linkcolor=linkcolor,urlcolor=urlcolor, colorlinks=true}

\begin{document}
    \begin{center}
        \hfill \break 
        \large{Санкт-Петербургский государственный университет} \\
        \large{Прикладная математика и информатика} \\
        \hfill \break
        \hfill\break
        \hfill \break
        \hfill \break
        \hfill \break      
        \large{Кизеев Данил Владимирович} \\
        \large{Группа №222} \\
        \hfill \break
        \hfill \break
        \hfill \break  
        \large{<<Построение кодов, исправляющих ошибки>>\ } \\
        \hfill \break
        \hfill \break
        \large{Курсовая работа}
    \end{center}
    \hfill \break
    \hfill \break
    \hfill \break
    \hfill \break
    
    \begin{flushright}
        \large{Научный руководитель:} \\
        \large{доктор ф.-м. наук Генералов Александр Иванович} \\
        \large{Кафедра высшей алгебры и теории чисел}\\
    \end{flushright}
    \hfill \break
    \hfill \break
    
    \begin{center}
        \large{Санкт-Петербург} \\
        \large{2019}\\
    \end{center}
    \thispagestyle{empty}   %- отображение номера снизу
    
    % КОНЕЦ ТИТУЛЬНОГО ЛИСТА
    \newpage
        \begin{center}
            \large \textbf{Введение}
        \end{center}
            \hfill \break
            \hfill \break
            В эпоху информационных технологий возникла необходимость передавать данные на большие расстояния. Не существует идеальных каналов связи, а значит, в передаваемых цифровых последовательностях существует вероятность появления ошибок.    
            
            Для борьбы с этими ошибками и были открыты классы кодов, контролирующих ошибки.
             
            Задача кодирования состоит в добавлении к информационным символам дополнительных с целью корректировки помех, возникающих в процессе передачи информационной последовательности.
            Таким образом, исходная последовательность представляется более длинной последовательностью для корректировки помех.
            
            Дадим определение данных цифровых последовательностей, или двоичных кодов(поскольку все последовательности построены из нулей и единиц).\\
            
            \textbf{Определение:}
            
            	Двоичный код мощности M и длины n представляет собой множество из M двоичных слов длины n, называемых кодовыми словами. Обычно M = $2^k$, где k - некоторое целое число; такой код называется двоичным (n, k)-кодом.\\
           
           	Первой задачей курсовой работы было построить кодер с вылавливанием ошибок для (19437, 19408) кодов Файра с порождающим многочленом $$g(x) = (x^{19} - 1) \times p(x),$$ где $p(x)$ - примитивный многочлен степени 10. \\ 
           	
           	Вторая задача заключалась в построении эффективных кодов с помощью перемежения и укорочения менее эффективных кодов из таблиц, представленных в книге Ричарда Блейхута "Теория и практика кодов, контролирующих ошибки". 
           	\\Параметры кодов соответственно:\\ 1. (1072, 1024)-код для исправления пакетов ошибок длины 16(циклический код).\\2. (1080, 1024)-код для исправления пакетов ошибок длины 20(код Файра).\\
           	
           	Каждая из задач была успешно выполнена.\\
           	Обе задачи были реализованы в виде компьютерных программ на языке C++. Использованная среда разработки - CLion 2019.1.
        
        
    \newpage
        \begin{center}
            \large \textbf {Описание выполненных задач}\\
            \large \textbf {Задача №1}\\
            \large \textbf {Создание кодера для кодов Файра}
        \end{center}
    
    	Для реализации кодера вылавливающего ошибки, была использована парадигма объектно-ориентированного программирования - для потенциального расширения программы в будущем.\\
    	
    	Для облегчения чтения и уменьшения объема отчёта, код был загружен на сайт \href{https://github.com}{github.com} и доступен по ссылке:
    	\href{https://github.com/breadfan/Crypto_proj}{github.com/breadfan/Crypto\_proj}.\\
       	\begin{center}
       		\large \textbf {Алгебраическая структура задачи}
        \end{center}	
        
        
         \hfill \break
        Мы будем рассматривать и работать с кодами, к которым добавляются избыточные проверочные разряды, предназначенные для исправления потенциально возможных ошибок. Такие коды(с избыточностью) называются корректирующими кодами и широко используются в криптографии. \\
        
        Для помехоустойчивых кодов лучше всего подходят циклические коды. Приведём строгое определение: \\
        
        \textbf{Определение:}
        
        Линейный код - это код, что множество их кодовых слов образует k-мерное линейное подпространство в n-мерном линейном пространстве, изоморфное пространству n-битовых векторов. Мы будем рассматривать подкласс линейных кодов - циклические коды, которые, в свою очередь, удовлетворяют следующему условию: при циклическом сдвиге кода последовательность снова будет кодовым словом. \\
        Определим понятие \textbf{порождающего} многочлена - это такие многочлены, которые используются для кодировки и декодировки исходных кодовых последовательностей и которые и создают, "порождают" кодовую последовательность. В нашем случае мы будем искать у каждого кода остаток от деления на этот самый "порождающий" многочлен и передавать его определённым образом вместе с изначальной последовательностью определённым образом.\\
        
        Для удобства мы будем рассматривать наши кодовые последовательности в виде полиномов:
        $$Q(x) = a_{n-1}x^{n-1} + a_{n-2}x^{n-2} + ... + a_{1}x + a_{0},$$
        где $a_{i}, i = \overline{0, (n-1)}$ - ноль или единица соответственно кодовой последовательности.
   		Данное представление даёт нам возможность свести операции с кодами к операциям с многочленами, причём все действия здесь выполняются в конечном поле $GF(2)$, то есть по модулю два. Для сложения двух полиномов нам нужно просто будет сложить по модулю два коэффициенты при одинаковых степенях этих многочленов. Для умножения же нужно перемножить степени и сложить по модулю два коэффициенты при одинаковых степенях.\\
   		
   		Идея построения кодов Файра основана на использовании неприводимых многочленов, то есть таких многочленов, которые нельзя представить в виде произведения многочленов меньших степеней.\\
   		
   		\textbf{Определение:}
   		
   		\textit{Кодом Файра} называется исправляющий пакеты ошибок циклический код над конечным полем $GF(q)$ с порождающим многочленом
   		$$g(x) = (x^{2t - 1} - 1)p(x),$$
   		где $p(x)$ - примитивный многочлен над $GF(q)$, степень  $m$ которого не меньше длины  $t$ исправляемого пакета и который не делит  $x^{2t - 1} - 1$. Длина $n$ кода Файра равна наименьшему целому $n$ такому, что  $g(x)$ делит $x^n - 1$.
   		
   		Также длину можно узнать из уравнения $$n = c(2^m - 1),$$где $c = 2t - 1$.
   		Число проверочных разрядов в каждом слове равно
   		$$r = c + m.$$ 
   		По условию нам дан порождающий полином $g(x) = (x^{19} - 1) \times p(x)$, где $p(x)$ можно найти из таблицы: это многочлен  $$p(x) = x^{10} + x^3 + 1,$$ тогда $$g(x) = (x^{19} - 1)(x^{10} + x^3 + 1) = x^{29} + x^{22} + x^{19} + x^{10} + x^3 + 1$$  и длина кода $$n = 19437.$$ Количество информационных символов: $$k = 19408.$$ Тогда количество проверочных символов $$r = n - k = 19437 - 19408 = 29.$$
   		Получим длину исправляемого пакета c = 2t - 1 = 9. Тогда t = 10.\\
   		Осталось проверить достаточность проверочных символов для исправления пакетов длиной t = 10: количество проверочных символов должно быть не менее, чем 3t - 1. При t = 10 и r = 29 это условие выполняется.\\  
   		
   		Приступим к разработке \textit{кодера}.\\
   		Введём в обозначения операцию $$R_{b(x)}[a(x)].$$ Для многочленов $a(x)$ и $b(x)$ - она будет означать остаток от деления $a(x)$ на $b(x)$.
   		Чтобы закодировать слово, нам нужно представить его в в виде $$c(x) = x^{n - k}i(x) + t(x),$$ где $i(x)$ - исходная информационная последовательность, а $t(x)$ выбирается так, чтобы выполнялось условие $$R_{g(x)}[c(x)] = 0.$$
   		Получим, что $$R_{g(x)}[x^{n - k}i(x)] + R_{g(x)}[t(x)] = 0$$ и  $$t(x) = -R_{g(x)}[x^{n - k}i(x)].$$
   		$t(x)$ будем называть \textit{синдромом} многочлена. Наконец, нам нужно сложить по модулю два $t(x)$ и $x^{n - k}i(x)$.
   		Таким образом, для \textit{кодирования} исходной информационной последовательности нам нужно приписать справа $r$ нулей, разделить эту последовательность на порождающий полином, узнав таким образом остаток, и приписать этот остаток на место добавленных нулей.
   		
   	\newpage
   		\begin{center}
   			\large \textbf {Техническая структура задачи}
   			
   			\large Весь код доступен по ссылке: \href{https://github.com/breadfan/Crypto_proj}{github.com/breadfan/Crypto\_proj}
   		\end{center}	
   		Программа была написана языке программирования C++ с использованием парадигмы объектно-ориентированного программирования: был создан класс "Polynomial" для расширения возможностей программы в будущем - создания бо$\acute{}$льшего множества объектов и работы с ними.
   		 
   	\newpage
   		\begin{center}
   			\textbf \large {Задача №2:}
   		\end{center} 
   		
   		Программа была написана на языке C++.\\
   		Для реализации эффективных кодов был использован ранее написанный класс "Polynomial".
   		\\
   		Конечный код был получен с помощью перемежения и укорочения циклических кодов. Далее приводится определение данных понятий.
   		
   		\begin{center}
   			\large \textbf {Алгебраическая структура задачи}
   		\end{center}	 
   		\textbf{1}. Построение первого кода с параметрами (1072, 1024) для исправления пакетов ошибок длины 16.
   		Построить данный код предлагалось, выбрав за основу один из кодов, приведённых в таблице 5.1 книги Р.Блейхута. Подбором из кодов данной таблицы было выяснено, что подходящим является код с параметрами (511, 499), длиной исправляемого пакета, равной 4 и порождающим многочленом $$x^{12} + x^{8} + x^5 + x^3 + 1.$$
   		Теперь определим понятия \textit{укороченных} циклических кодов и \textit{перемежения} их же.\\
   		
   		Любой циклический код можно \textbf{укоротить}, т.е. от (n, k) кода перейти к (n - b, k - b) коду выбрасыванием b информационных позиций в каждом кодовом слове. Будем полагать, что выбрасываются b старших разрядов кода и b < k.
   		Сами выбрасываемые символы положим равными нулю и передавать не будем. На декодере это никак не скажется - он будет работать по полной кодовой последовательности, в начале своей работы полностью её восстанавливая.\\
   		
   		Важно заметить, что укороченный циклический код уже не является циклическим, так как циклический сдвиг кодовой последовательности не всегда будет являться кодовой последовательностью. \\
   		
   		Чтобы из $(n, k)$-кода получить $(jn, jk)$-код, нужно выбрать из исходного кода $j$ произвольных слов и укрупнить кодовые слова, чередуя их символы каждой последовательности по очереди. Так, получим, что, если исходный код исправлял пакет ошибок длиной $t$, то укрупнённый код будет исправлять пакеты ошибок длиной $jt$.
   		Чтобы получить код с помощью перемежения, нужно заменить порождающий многочлен $g(x)$ исходного кода на $g(x^j)$ и пересчитать многочлен, умножив полученное в результате перемежения слово на порождающий многочлен $g(x^j)$.\\
   		
   		Чтобы осуществить метод перемежения, который увеличивает код в $j$ раз, нужно изначально $j$ сгенерированных слов, принадлежащих исходному коду.
   		В нашем случае исходный $(511, 499)$-код(все $j$ = 4 слов) был сокращен на $b = 243$ до кода $(268, 256)$.\\
   		Далее, с помощью добавления синдромного многочлена код был модифицирован до кода, исправляющего ошибки. Далее у каждого кодового слова были отрезаны $243$ бита последующим перемежением до $(1072, 1024)$-кода. В свою очередь, перемежение было организовано так: из четырёх последовательностей кода составляется одна большая последовательность, полученная чередованием битов каждого из 4 слов. Если всё сделано правильно, алгоритм на этом заканчивается - код готов. 
   		
   		
   		\textbf{2}. Построение второго кода с параметрами $(1080, 1024)$ для исправления пакетов ошибок длины 16. 
   		Не найдя подходящего кода в таблице $5.1$, я обратился к следующей таблице - $5.2$ и нашёл там подходящий код.
   		Этим кодом являлся $(279, 265)$-код, исправляющий пакеты ошибок длиной 5.\\
   		Он был укорочен на $b = 9$ до $(270, 259)$-кода, а далее перемежением, подобно предыдущему случаю для (1072, 1024)-кода, был модифицирован до (1080, 1024)-кода.
   		
   		
   		 
   		\begin{center}
   			\large \textbf {Технические особенности задачи}
   		\end{center}
   	
   		Для удобства работы с большими кодовыми последовательностями, в качестве контейнера для кодов используется псевдоконтейнер std::string.\\	 
   		Сами коды задаются с помощью метода random() из библиотеки <iostream>, также доступен ввод с клавиатуры.  
   		 
   		 
   		\newpage
   		\begin{center}
   			\large \textbf {Использованные источники информации}
   		\end{center}
   		Блейхут Р. Теория и практика кодов, контролирующих ошибки[Текст]/Р.Блейхут; пер. с анг. И.И.Грушко и В.М.Блиновского, ред. К.Ш.Зигангирова. - Москва, изд.$"$Мир$"$ , 1986. - 576 с.
   		\\
   		
   		
   		У.Питерсон, Э.Уэлдон. Коды, исправляющие ошибки[Текст]/У.Питерсон, Э.Уэлдон; ред Р.Л.Добрушиной и С.И.Самойленко - Москва, изд.$"$Мир$"$, 1976 - 594 с.
        
        
\end{document}
